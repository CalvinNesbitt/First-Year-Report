\section{Lyapunov Analysis of Some Simple Chaotic Models} \label{section: Lyap-Analysis}

Lyapunov exponents decribe the asymptotic time evolution of phase space elements, in particular, the average rate at which they expand or contract. They are used to charactise chaotic dynamics and are widely used in the analysis of nonlinear dynamical systems \cite{Beck1995}. In analogy with an eigenvalue/eigenvector problem, there also exist vectors corresponding to these exponents that contain information on the directions in which phase space elements stretch or shrink. These so called covariant Lyapunov vectors (CLVs) are tangent directions that point along stable/unstable manifolds at each point in phase space and are required if one wishes to give a complete characterisation of a chaotic dynamical system \cite{Pikovsky2016a}.\\

Whilst Lyapunov exponents represent asymptotic or average properties, inhomogenities in the phase space of chaotic systems create fluctuations in the stability properties of different parts of the attractor \cite{Politi2014}. Understanding such fluctuations can be done through the investigation of finite time Lyapunov exponents (FTLEs), something which we have looked at from a large deviations perspective \cite{Touchette2009}.\\

In section \ref{subsection: LE Basics} we will outline the underlying mathematical theory of Lyapunov exponents and CLVs. In section \ref{section: FTLE-Fluctuations} we will then explain the aforementioned large deviation approach to the fluctuations in the attractors stability. This lays the ground for sections \ref{subsection: L63 Results} and \ref{subsection: L96 Results} where we discuss our finding on the application of these methods in the L63 and L96 models respectively. Finally, in section \ref{subsection: Lyapunov future} we will discuss our expectations for the future directions of this work.
