\subsection{Fluctuations of Lyapunov Exponents} \label{section: FTLE-Fluctuations}

Beyond simply calculating the Lyapunov exponents and CLVs we have also been interested in understanding the statistics of the FTLEs. Following \cite{Politi2014} this is of importance for understanding inhomogeneity in a strange attractor. In particular the stability properties throught the attractor are reflected in the FTLE fluctuations. As noted in \cite{Politi2014} as the finite averaging time $\tau$ used to define each FTLE in \ref{FTLE definition} is increased, the FTLE time series will fluctuate less and less around its corresponding mean value (the global LE $\lambda$). Moreover, we can invoke the theory of large deviations to propose the existence of a rate function $I$ in the large $\tau$ limit

\begin{align}
    I(\lambda) = \lim _{\tau \to \infty} - \frac{P(\lambda, \tau)}{\tau} \label{rate}
\end{align}

where $P(\lambda, \tau)$ is the pdf of the FTLEs averaged over a time $\tau$.\\

In both the Lorenz 63 and 96 models we have approximated $P(\lambda, \tau)$ using a simple kernel density estimation technique and then used \ref{rate} to calculate the rate function $I$ directly.
