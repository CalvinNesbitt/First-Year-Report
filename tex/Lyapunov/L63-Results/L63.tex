\subsection{Lorenz 63} \label{subsection: L63 Results}

The 1963 model of Edward Lorenz (L63) is a widely studied $3$ dimensional set of ordinary differential equations that capture some of the unpredictabile behaviour typically associated with atmospheric motions \cite{Lorenz1963} \cite{Sparrow1982}. As a first step towards calculating the Lyapunov vectors and exponents in higher dimensional models we have studied them in L63. In particular we have calculated the full spectrum of Lyapunov exponents and vectors that were discussed in section \ref{subsection: LE Basics}. Moreover we have investigated the angles between the CLVs and attempted to demonstrate numerically the existence of large deviation laws for the FTLEs as outlined in section \ref{section: FTLE-Fluctuations}.\\

\begin{figure}
\centering
\includegraphics[width=0.6\textwidth, keepaspectratio]{Theta-Density.pdf}
\caption{The effect of the parameter $c$ on the Lyapunov exponents of Lorenz $96$. A larger value of $c$ corresponds to less near $0$ exponents as well as an increase in the absolute value of the most positive and negative exponents.}
\label{fig:c-effect-CLE}
\end{figure}
