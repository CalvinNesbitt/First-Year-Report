\subsection{Lyapunov Analysis of the Lorenz 63 Model} \label{subsection: L63 Results}

The 1963 model of Edward Lorenz (L63) is a widely studied $3$ dimensional set of ordinary differential equations that are known to capture some of the unpredictabile behaviour typically associated with atmospheric motions  \cite{Lorenz1963} \cite{Sparrow1982}:


\begin{align}
    \dv{x}{t} &= \sigma (y - x), \\
    \dv{y}{t} &= x(\rho - z) - y,\\
    \dv{z}{t} &= xy - \beta z.
\end{align}

Here $\sigma$, $\rho$ and $\beta$ are free parameters. As a first step towards calculating the Lyapunov vectors and exponents in higher dimensional models we have studied them in L63. In particular we have calculated the full spectrum of Lyapunov exponents and vectors that were discussed in section \ref{subsection: LE Basics} for a classical parameter choice where the system is known to be chaotic. Moreover we have investigated the angles between the CLVs with the aim of understanding the partial hyperbolicity of the system. As an example result in this direction, in figure \ref{fig:L63-Angles} we have reproduced a result of Kuptsov \cite{Kuptsov2018} demonstrating that the L63 model is partially hyperbolic, namely one can partition the tangent space in to a non-contracting space and a contracting space. Finally, we have also attempted to demonstrate numerically the existence of large deviation laws for the FTLEs as outlined in section \ref{section: FTLE-Fluctuations}. Whilst promising progress has been made in this direciton we need to account for correlations in the FTLE timeseries as discussed in \cite{Galfi2019} before we can satisfactorally say that convergence has been acheived.

\begin{figure}
\centering
\includegraphics[width=0.6\textwidth, keepaspectratio]{Theta-Density.pdf}
\caption{First note in L63 there exist a unstable, centre and stable CLV. We denote by $\theta _1$ the minimal angle between the plane spanned by the unstable/centre CLVs and the line spanned by the stable CLV. By $\theta _2$ the minimal angle between the plane spanned by the stable/centre CLVs and the line spanned by the unstable CLV. The density of both $\theta _1$  and $\theta _2$ are both displayed. The important take away is that $theta_1$ is bounded away from $0$ so the unstable-centre space is isolated from the stable space.}
\label{fig:L63-Angles}
\end{figure}
