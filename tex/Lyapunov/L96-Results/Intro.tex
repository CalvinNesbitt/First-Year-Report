\subsection{Lyapunov Analysis of the Two Scale Lorenz $96$ Model} \label{subsection: L96 Results}

The two scale Lorenz 96 model mimics the evolution of two scale separated atmospheric variables on a latitudinal ring. Following \cite{Carlu2019} we write it explicitly as

\begin{align}
\dv{x_k}{t} &= x_{k-1} \left(x_{k+1} - x_{k-2} \right) -x_k + F_s - \frac{hc}{b} \Sigma_{j} y_{k, j} \label{slow-variables}\\
\dv{}{t} y_{k,j} &= cb y_{k, j+1} \left(y_{k, j-1} - y_{k, j + 2} \right) - c y_{k, j} + \frac{c}{b} F_f + \frac{hc}{b} x_k. \label{fast-variables}
\end{align}

Here \ref{slow-variables} represents the time evolution of a spatially large, slow variable that is designed to model a synoptic scale process whilst \ref{fast-variables} represents the evolution a smaller, fast variable that models convective motions.\\

Building on out work in L63 we have also calculated Lyapunov exponents, vectors and associated rate functions for the L96 model. Moreover we have investigated the effect of the time scale seperation parameter $c$ and the coupling strength paraemter $h$ on the aforementioned quantities. Three specific questions we sought to address were:

\begin{enumerate}
    \item Is it possible to associate large/small scale processes with particular CLVs? This was inspired by the findings \cite{Vannitsem2016}. \label{question: scale}
    \item Can we obtain large deviation principles for the FTLEs? If so, are the rate functions for the FTCLEs and FTBLEs the same? This question was inspired by the work in \cite{Vannitsem2016} and \cite{DeCruz2018}. \label{question: l96-LDP}
    \item To what extent are the stable, unstable and centre manifolds distinguishable from one another? This will help us address the question of whether the Lorenz $96$ can be modelled as hyperbolic or not. This largely builds on \cite{Carlu2019} \label{question: L96-hyperbolic}.
\end{enumerate}

Following \cite{Vannitsem2016} we found looking at the CLV variance provided an affirmative answer to question \ref{question: scale}. In particular we found that the CLVs associated to near $0$ Lyapunov exponents had positive correlations with the synoptic scale variables whilst those large in magnitude, both positive and negative were associated with convective motions. This is what one would intuitively expect given that the convective motions take place on much shorter timescales.\\

Our answer to \ref{question: l96-LDP} was inconclusive since the length of timeseries needed to compute the rate function by brute force proved impractical. In answer to \ref{question: L96-hyperbolic} we found a large projection on to the centre direction that we associated with slow scale processes. Indeed in the limit of the time scale seperation parameter one only finds center CLVs associated to the synoptic variables.
