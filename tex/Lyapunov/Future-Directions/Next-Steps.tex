\subsection{Future Directions} \label{subsection: Lyapunov future}

% There are three directions in which we aim to build upon this work.\\

The first has already begun, and will be explained further in section \ref{section: LRT}: the use of CLVs in linear response theory.\\

% A second is inspired by the recent work of Galfi \cite{Galfi2019} which investigates spatiotemporal large deviations.\\

A currently more speculative direciton is that of unstable periodic orbits (UPOs). UPOs are robust topological features of strange attractors that are experimentally availaible \cite{Cvitanovic1988}. Through the use of so called `trace formulas' one is able to approximate the asymptotic average of measurable observables with respect to the invariant measure of a dynamical system. Interestingly, the weights in such formulas are dependent on knowledge of the Lyapunov exponents whilst the UPOs themselves can be used to obtain the Lyapunov exponents \cite{Pikovsky2016} \cite{Cvitanovic}. These are areas we intend to learn more about in the coming year, in the hope of collaborating with fellow MPE student Chiara Maiocchi. A particular hope would be to associate specific UPOs and CLVs to coupled oceanic-atmospheric modes in the MAOOAM model \cite{DeCruz2016}. This is motivated by a recent study of UPOs to understand atmospheric blocking \cite{Lucarini2020}.
