\subsection{The Split Sensitivity Algorithm}

Recently a numerically efficient reformulation of \ref{eqn: Ruelle Formula} has been proposed by Chandramoorthy and Wang \cite{Chandramoorthy2020}. This so called `Space-Split Sensitivity' or `s3' algorithm is based on splitting the perturbation vector field $\epsilon(x)$ of \ref{eqn: Ruelle Formula} along the stable and unstable directions in tangent space. These directions are precisely those subspaces $E_i$ in \ref{Ruelle Split} corresponding to strictly negative and positive Lyapunov exponents respectively.\\

One slight limitation of the algorithm as presented in \cite{Chandramoorthy2020} is that it is only applicable to maps, and does not consider the centre direction in the case of flows. We are currently in the process of implementing this for the aforementioned Lorenz $63$ system and are in contact with the authors of \cite{Chandramoorthy2020} who are working on this problem also. Our hope following the application to L63 is to apply the alogrithm in higher dimensional models that are relevant in geophysical fluid dynamics such as MAOOAM \cite{DeCruz2016}, a climate quasi-geostrophic model, which allows for a great flexibility of configurations and is able to describe both atmospheric and oceanic dynamics in a simplified yet meaningful way. Should the above fail, an alternative path may be that of shadowing that has been recently suggested \cite{Ni2020} \cite{Ni2020a}.  
